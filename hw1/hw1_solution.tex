% !TEX program = XeLaTeX
% !TEX encoding = UTF-8
\documentclass[UTF8,nofonts]{ctexart}
\usepackage{fullpage,nicefrac,amsmath}
\usepackage{hyperref}
\setCJKmainfont[BoldFont=STHeiti,ItalicFont=STKaiti]{STSong}
\setCJKsansfont[BoldFont=STHeiti]{STXihei}
\setCJKmonofont{STFangsong}

\newcommand{\N}{\mathbb{N}}
\newcommand{\Z}{\mathbb{Z}}
\newcommand{\Q}{\mathbb{Q}}
\newcommand{\R}{\mathbb{R}}

\begin{document}
\section*{Performance Evaluation HW1}
  \noindent
  姓名:鍾毓安         \\
  系級:資訊四 	     \\
  學號:B01902040     \\
  繳交期限:2016.3.23
  \begin{enumerate}
    \item{
      I think the statement is wrong.
      If it is true, then exactly the same situation (expectation value equals to 1.25Y) will happen after we decide to swap to another door, this means that the action of swapping could last forever.
      A much more reasonable explanation is as follows.
      Assume we choose door A, then the expectation value of A is $\frac{1}{2} \cdot x + \frac{1}{2} \cdot 2x = 1.5x$.
      The expectation value of B, which is another door that may contains twice or half of money of A, is still $\frac{1}{2} \cdot x + \frac{1}{2} \cdot 2x = 1.5x$, which is equals to that of A.
      Therefore, it doesn't matter whether we swap the door or not.
      Reference: \url{https://en.wikipedia.org/wiki/Two_envelopes_problem}
    }
    \item{
      Assume box A is the box that contains two red balls, box B is the box that contains two blue balls, and box C is the box that contains a red ball and a blue ball.
      \begin{enumerate}
        \item{
          P(\text{The second ball is also blue. | The first ball is blue.}) 
          \begin{eqnarray*}
            &=& \frac{P(\text{Both the first and second balls are blue.})}{P(\text{The first ball is blue.})} \\
            &=& \frac{\frac{1}{3}}{\frac{1}{3} + \frac{1}{3}\cdot \frac{1}{2}} = \frac{2}{3}
          \end{eqnarray*}
        }
        \item{
          P({\text{The second ball is also blue. | There is one blue ball in the chosen box.}})
          \begin{eqnarray*}
            &=& \frac{P(\text{Both balls are blue.})}{P(\text{There is one blue ball in the chosen box.})} \\
            &=& \frac{\frac{1}{3}}{\frac{2}{3}} = \frac{1}{2}
          \end{eqnarray*}
        }
      \end{enumerate}
    }
    \item{
      $X_{i}$ are i.i.d. exponentially distributed $\Rightarrow f_{X_{i}}(x) = \lambda e^{-\lambda x}, \lambda \geq 0$.
      \begin{eqnarray*}
        F_{X_{i}}^{*}(s) &=& \int_{0}^{\infty}\lambda e^{-\lambda x}e^{-sx}dx \\
        &=& \int_{0}^{\infty}\lambda e^{-(s + \lambda) x}dx \\
        &=& \frac{\lambda}{s + \lambda}[-e^{-(s + \lambda)x}]_{0}^{\infty} \\
        &=& \frac{\lambda}{s + \lambda}
      \end{eqnarray*}
      $\tilde{N}$ is geometrically distributed $\Rightarrow P_{\tilde{N}(n = i)} = p(1 - p)^{i - 1}$.
      \begin{eqnarray*}
        \tilde{N}(z) &=& \sum_{i=1}^{\infty}p(1 - p)^{i - 1}z^{i} \\
        &=& \frac{zp}{1 - (1 - p)z}
      \end{eqnarray*}
      \begin{eqnarray*}
        Y^{*}(s) &=& \tilde{N}(F_{X_{i}}^{*}(s)) \\
        &=& p \cdot \frac{F_{X_{i}}^{*}(s)}{1 - (1 - p)F_{X_{i}}^{*}(s)} \\
        &=& \frac{p \cdot \frac{\lambda}{s + \lambda}}{1 - (1 - p)\frac{\lambda}{s + \lambda}} \\
        &=& \frac{\lambda p}{s + \lambda p}
      \end{eqnarray*}
      \begin{enumerate}
        \item{
          $E(Y) = [-\frac{dY^{*}(s)}{ds}]_{s=0} = [-\frac{-\lambda p}{(s + \lambda p)^2}]_{s=0} = \frac{1}{\lambda p}$
        }
        \item{
          $Var(Y) = [\frac{d^{(2)}Y^{*}(s)}{ds^2}]_{s=0} = [\frac{2\lambda p}{(s + \lambda p)^3}]_{s=0} = \frac{2}{(\lambda p)^2}$
        }
      \end{enumerate}
    }
  \end{enumerate}
\end{document}
